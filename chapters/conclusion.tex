\chapter{Conclusion: the Future}
\label{ch:conclusion}

The field of automatic programming has progressed a lot in the timespan of the writing of this thesis\footnote{ensuring a very fun and totally unstressful writing experience}.
Major topics of contention shifted from "can program synthesis be extended from logic and functional to imperative programming languages?" \cite{polikarpovaStructuringSynthesisHeapmanipulating2019} to "should software developers read the code they deploy?" \cite{andrejkarpathy[@karpathy]TheresNewKind2025}.
Competitive programming is at a historical stage at which chess was in 1996 \cite{pandolfiniKasparovDeepBlue1997} where the top 0.2\% human competitors still outperform the top algorithms \cite{openaiCompetitiveProgrammingLarge2025}.
Part \ref{part:proginduction} traces these developments like a fossil record, from domain specific languages in chapters \ref{ch:bfpp} and \ref{ch:neurogen} to test time scaling of language models in chapters \ref{ch:boptest} and \ref{ch:seidr}, advancing the state of the art in both areas.
It concludes with introduction and validation of \emph{Synthesize Execute Debug Instruct Rank} - a program synthesis framework that introduces testing into language model driven program synthesis for the benefit of safety-critical domains such as healthcare.

Part \ref{part:health} advances the state of the art in patient simulators and demonstrates that in conjunction with such a simulator, SEIDR can be used to automatically discover treatment strategies for healthcare settings and often do so \emph{better than traditional reinforcement learning methods}, in addition to scientific discovery potential of interpretable strategies.

At the same time, our headline benchmark for Reinforcement Learning from Code Execution in Patient Simulators, Auto-ALS, remains unsolved, raising an important question: \emph{what's the missing ingredient?}

This question shall be subject of future research, as the research body of this thesis does not support anything more than an educated guess.
Namely, what if the missing ingredient is simply a more advanced language model?
Given SEIDR's ability to get much more out of a language model than naive sampling would suggest possible and the never-ending progress of models, it is reasonable to assume that SEIDR with the latest model will soon solve Auto-ALS and could be used to discover treatment strategies for datasets such as MIMIC-SEQ.

In other words, I regret to inform you that this thesis has not solved healthcare.
But one can be reasonably certain that RLCEPS will play a role in the future of healthcare and justifiably hopeful that program synthesis methods introduced in chapters \ref{ch:bfpp}-\ref{ch:seidr} and patient simulators proposed in chapters \ref{ch:heartpole}-\ref{ch:mimicseq} will play a role in this transition.

\emph{Finis tantum principium est}.