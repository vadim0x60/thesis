\chapter{Conclusion: the Future}
\label{ch:conclusion}

The field of automatic programming has progressed a lot in the timespan of the writing of this thesis\footnote{ensuring a very fun and totally unstressful writing experience}.
Major topics of contention shifted from "can program synthesis be extended from logic and functional to imperative programming languages?" \cite{polikarpovaStructuringSynthesisHeapmanipulating2019} to "should software developers read the code they deploy?" \cite{andrejkarpathy[@karpathy]TheresNewKind2025}.
Competitive programming is at a historical stage at which chess was in 1996 \cite{pandolfiniKasparovDeepBlue1997} where the top 0.2\% human competitors still outperform the top algorithms \cite{openaiCompetitiveProgrammingLarge2025}.
Part \ref{part:proginduction} traces these developments like a fossil record, from domain specific languages in chapters \ref{ch:bfpp} and \ref{ch:neurogen} to test time scaling of language models in chapters \ref{ch:boptest} and \ref{ch:seidr}, advancing the state of the art in both areas.

However, application of program synthesis in safety critical domains such as Healthcare as discussed in chapter \ref{ch:promise} remain rare.
Chapter \ref{ch:proposal} proposed PatientSPIRL: a framework for improving quality of care with program synthesis technology.
Chapters \ref{ch:heartpole}-\ref{ch:imagym} introduced a suit of novel benchmarks to train and validate PatientSPIRL before it can be deployed in real life settings.

Finally, with these foundations laid, chapter \ref{ch:eval} presented a PatientSPIRL prototype for HeartPole and Virtu-ALS benchmarks that shows both the great promise of the approach and where it currently falls short.
Based on these results, one cannot claim that deep program induction for personal healthcare has been \emph{solved}.
It is not even certain that state-of-the-art methods of \emph{programmatically interpretable reinforcement learning} are as of the date of present publication fully ready to be deployed in such sensitive contexts.
But one can be reasonably certain that they will play a role in the future of healthcare and justifiably hopeful that advances introduced in chapters \ref{ch:bfpp}-\ref{ch:seidr} and \ref{ch:heartpole}-\ref{ch:imagym} will play a role in this transition.

\emph{Finis tantum principium est}.