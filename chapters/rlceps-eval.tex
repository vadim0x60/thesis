\chapter{Evaluating RLCEPS}
\label{ch:rlceps-eval}

Now that we've firmly established the background, we can proceed to the topic of this thesis. 
Namely, implementing a proof-of-concept for Reinforcement Learning from Code Execution in Patient Simulators.
This was intentionally avoided in the previous chapters, since, while we did find both existing program induction methods (see chapter \ref{ch:ps-methods} and existing patient simulators (see chapter \ref{ch:simulators-sota}) inadequate for the purposes of RLCEPS, testing novel methods on novel benchmarks would render it effectively impossible to meaningfully realate the results to the state of the art.


We run a program synthesis experiment using SEIDR (see chapter \ref{ch:seidr}) with hyperparameters as desribed in table \ref{tab:rlceps-auto-als}

\begin{table}
    \centering
    \begin{tabular}{|c|c|c|c|c|c|}
        model & $\treearity_\text{draft}$ & $\treearity_\text{explain}$ & $\treearity_\text{debug}$ & $\beamwidth$ & selection \\
        \midrule
        gpt-4o & 3 & 2 & 2 & 5 & tournament
    \end{tabular}
    \caption{Hyperparameter choice for evaluating RLCEPS on Auto-ALS}
    \label{tab:rlceps-auto-als}
\end{table}

\newpage
\section{HeartPole results}
\todo{Turn this into actual text}

\paragraph{Prompts}

\paragraph{Generated program}

With a Deepseek model:

\lstinputlisting{listings/metanurse-deepseek.py}

With GPT:

\lstinputlisting{listings/metanurse-gpt.py}

for 2868 iterations

\paragraph{Evaluation}

\begin{tabular}{lrr}
\toprule
model & avg score & best avg score \\
\midrule
deepseek-coder & -149.952560 & 0.400517 \\
gpt-4o & -99.885775 & 0.348758 \\
\bottomrule
\end{tabular}

\newpage
\section{Auto-ALS results}

\paragraph{Prompts}

Consider the summary table of the ABCDE protocol:

\begin{lstlisting}
+-----------------+---------------------------------------------------+
| Assessment      | Treatment                                         |
+=================+===================================================+
| A – Airways     | Voice                                             |
|                 | Breath sounds                                     |
|                 | Oxygen (15 l min−1) Suction                       |
+-----------------+---------------------------------------------------+
| B – Breathing   | Respiratory rate (12–20 min−1)                    |
|                 | Chest wall movements                              |
|                 | Chest percussion                                  |
|                 | Lung auscultation                                 |
|                 | Pulse oximetry (97%–100%)                         |
+-----------------+---------------------------------------------------+
|                 | Seat comfortably                                  |
|                 | Rescue breaths                                    |
|                 | Inhaled medications                               |
|                 | Bag-mask ventilation                              |
|                 | Decompress tension pneumothorax                   |
+-----------------+---------------------------------------------------+
| C – Circulation | Skin color, sweating                              |
|                 | Capillary refill time (<2 s)                      |
|                 | Palpate pulse rate (60–100 min−1)                 |
|                 | Heart auscultation                                |
|                 | Blood pressure (systolic 100–140 mmHg)            |
|                 | Electrocardiography monitoring                    |
+-----------------+---------------------------------------------------+
|                 | Stop bleeding                                     |
|                 | Elevate legs                                      |
|                 | Intravenous access                                |
|                 | Infuse saline                                     |
+-----------------+---------------------------------------------------+
| D – Disability  | Level of consciousness – AVPU                     |
|                 | Alert                                             |
|                 | Voice responsive                                  |
|                 | Pain responsive                                   |
|                 | Unresponsive                                      |
|                 | Limb movements                                    |
|                 | Pupillary light reflexes                          |
|                 | Blood glucose                                     |
+-----------------+---------------------------------------------------+
|                 | Treat Airway, Breathing, and Circulation problems |
|                 | Recovery position                                 |
|                 | Glucose for hypoglycemia                          |
+-----------------+---------------------------------------------------+
| E - Exposure    | Expose skin                                       |
|                 | Temperature                                       |
+-----------------+---------------------------------------------------+
|                 | Treat suspected cause                             |
+-----------------+---------------------------------------------------+
\end{lstlisting}

Due to the inherent stochasticity of the environment (see chapter \ref{ch:auto-als}) we use 2 tests, which are 2 identical copies of the Auto-ALS environment.
Since lexicase selection assumes multiple deterministic tests, tournament selection is used instead.
We add hard limits on the length of the episode in terms of time (3 hours) as well as steps (350 environment steps).
The initial prompt is as follows:

\lstinputlisting{listings/metanurse_prompt.txt}

and, on iteration 37, the following snippet emerges, that will not be beaten until the end of the experiment (4697 iterations):

\lstinputlisting{listings/auto-als.py}
