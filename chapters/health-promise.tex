\citeself{section}{liventsevPhilHumansBenchmarkingMachine2024}

% Importance of Smart Healthcare
Understaffing has been consistently identified as the major challenge facing Healthcare today \cite{ashleyy.metcalfHospitalUnitUnderstaffing2016,SurveyShowsHidden1993,UnderstaffingSignificantIssue2012,campbellUniversalHealthCoverage2013, hudsonUnderstaffing2015, mercerMessageEditorinChief2008, r.stanleyUnderstaffedOverwhelmed2010, munnUnderstaffingWardsCompromising2017, thelancetHealthcareSystemStaffing2018}. Automation tools that make use of Machine Learning (also known as Healthcare 4.0 \cite{tortorellaHealthcareTrendsChallenges2020}) have been consistently identified as crucial for reducing the workload of Healthcare professionals and improving the quality of care \cite{agrawalMachineLearningHealthcare2020, deviDesignImplementationAdvanced2022, g.kumarSurveyMachineLearning2016, ganguliMachineLearningPursuit2020, maityMachineLearningImproved2017, mitraMachineLearningHealthcare2021, pianykhImprovingHealthcareOperations2020, xhaferraRoleMachineLearning2022}. In turn, the shortage of standard benchmarks has been consistently identified as a central roadblock for machine learning in Healthcare \cite{Crown2015Potential, David2020Evaluating, guSupervisedLearningPervasive2023, harutyunyanMultitaskLearningBenchmarking2019, Kathrin2022Benchmark, liventsevEffectivePatientSimulators2021, mcdermottReproducibilityMachineLearning2021, purushothamBenchmarkingDeepLearning2018, thelancetHealthcareSystemStaffing2018}.

Whether it's ImageNet \cite{dengImagenetLargescaleHierarchical2009} in Computer Vision or GLUE \cite{wangGLUEMultitaskBenchmark2018} in natural language processing, benchmarks are a core research tool in mature applications of machine learning, enabling quantitative analysis of learning methodologies to guide and orient their development.
Machine learning for Healthcare, an emergent field with unique challenges in availability of research datasets \cite{Anshik2021Handling, Gilbert2015market, Pahwa2021Big, Yazhini2019State} lacks an accepted benchmarking standard: recent literature reviews \cite{palMachineLearningHealthcare2023,tortorellaHealthcareTrendsChallenges2020} of the field cover a variety of studies that each use their own (often non-public) benchmark.

The lack of benchmarks is a crucial barrier to advancing the quality and accessibility of care with automatic programming. 
The most promising approach for advancing Healthcare with program synthesis is
\begin{enumerate}
    \item Use clinical data, such as electronic health records, to train a patient simulator: a predictive model of patient’s future health conditional on clinicians’ decisions. 
    \item Use Programmatically Interpretable Reinforcement Learning to search for a program/protocol that performs well in that simulator.
\end{enumerate}

Having developed a sufficiently robust PIRL algorithm, let us now turn our attention to part (1) - effective patient simulators.