\paragraph{Abstract}

One of the growing areas of computational intelligence is automatic programming, where a learning algorithm produces executable software. This paradigm promises efficient and highly capable artificial intelligence agents that express their knowledge and reasoning process in a programming language that can be understood, audited and edited by humans, as well as other automated tools - essential requirements for decision support systems in safety-critical application areas such as healthcare.

Of particular interest for healthcare is Programmatically Interpretable Reinforcement Learning, in which a program induction algorithm is used to search for a protocol that performs well in a predictive patient simulator.
The simulator can be derived from clinical data, implemented based on expert knowledge or combine both methods.
Deployment of this approach in real world settings is hindered by the lack of specialized patient simulators and insufficient capabilities of modern program synthesis algorithms.
This work makes contributions to both fields.

The contributions to synthesis algorithms are a novel programming language for general purpose neural program synthesis, a neurogenetic programming framework for program synthesis in BF++ or similar simple languages, a tree variational autoencoder model for code and, finally, Synthesize Execute Debug and Rank: a state-of-the-art iterative algorithm for fully autonomous programming with large language models.

In the field of patient simulators, an anthropodidactic Reinforcement Learning environment for emergency care (Auto-ALS) is introduced, a framework for image-based sonography simulators and a benchmark dataset for intensive care simulation. 

The proposed program synthesis algorithms are evaluated on standard benchmarks as well as Auto-ALS to identify healthcare-specific insights.
The advances introduced in this work lay the foundations for the nascent field of Programmatically Interpretable Reinforcement Learning for Healthcare.