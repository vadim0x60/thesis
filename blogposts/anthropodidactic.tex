\section{Anthropodidactic learning: a modest proposal}

\emph{Anthropodidactic machine learning} is using didactic materials developed for human students (textbooks, lectures and/or lecture notes, explanations, homeworks, exercises, \href{http://www.virtu-als.com/}{games} and other sorts of interactive edutainment) to train artificial intelligence.
Examples of anthropodidactic learning include using language textbooks to train a machine translation model or using a flight simulator developed for pilot training to train an autopilot with reinforcement learning \cite{staudingerXPlaneMLEnvironmentLearning2018}.

\paragraph{Motivation}

The education industry puts a lot of effort into curating and systematizing knowledge in ways that can be reasonably expected to be useful for a learner of any biological substrate
For example: 
\begin{itemize}
    \item Exercise sets in mathematics, physics and language learning, to name a few fields, are explicitly designed to cover all important clusters/corner cases of the subject area, something that isn't guaranteed in most datasets like logs, business records or text corpora.
    \item Exercise sets tend to be sorted by difficulty. This creates a useful curriculum to follow when training a machine learning model.
    \item Educational software aims to give users immediate and precise feedback on their mistakes: delayed gratification (also known as \emph{temporal credit assignment problem}), as it turns out, is hard for people \cite{tobinDelayGratificationReview2010} and reinforcement learning algorithms alike.
\end{itemize}

See \cite{brownArtProblemPosing2005} for more on the art of problem posing.

\paragraph{Related Work}

Machine learning community is undoubtedly interested in taking lessons from human learning, efforts to do so bear the umbrella term of \emph{antropomorphic
machine learning} \cite{angelovAnthropomorphicMachineLearning2018}. The prime example is curiculum learning \cite{sovianyCurriculumLearningSurvey2022, zhouCurBenchCurriculumLearning2024}: it was born with the observation that the order in which data is presented to human students is crucial for them achieving their learning goals, so it is likely a difference for machines as well.

However, examples of directly reusing learning aids developed for human students are hard to come by. A notable exception is Reinforcement Learning where decision-making agents are often trained on games initially intended for people. And while the claim that \emph{Atari games} \cite{mnihPlayingAtariDeep2013} and 
\emph{Minecraft} \cite{hofmannMinecraftAIPlayground2019} are educational material may be somewhat stretching the definition of education, interactive simulators first developed for people and later adapted for reinforcement learning include \emph{X-plane} \cite{staudingerXPlaneMLEnvironmentLearning2018} (used for training pilots) and Virtu-ALS (used for training nurses), to be introduced in section \ref{sec:auto-als}. 
Some antropodidactic work has also been done in natural language processing, training language models on children's books \cite{mayhewSimultaneousTranslationParaphrase2020} and exercises for language learning from \emph{Duolingo} \cite{mayhewSimultaneousTranslationParaphrase2020}

More recently, curated datasets with a focus on educational materials like FineWeb-EDU \cite{penedoFineWebDatasetsDecanting2024} have been used to improve the quality of large language model pretraining.

\paragraph{Conclusion}

In general, however, anthropodidactic learning remains underexplored. 
Didactic materials are a large class of useful data waiting for someone to turn them into a successful artificial intelligence system/product.
In this chapter, we apply the idea of anthropodidactic learning to patient simulation and in secion \ref{sec:auto-als} propose \emph{Auto-ALS}