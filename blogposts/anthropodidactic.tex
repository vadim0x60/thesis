\section{Anthropodidactic learning: a modest proposal}


\subsection{What is anthropodidactic learning?}

\emph{Anthropodidactic machine learning} is using didactic materials
developed for human students (textbooks, lectures and/or lecture notes,
explanations, homeworks, exercises, \href{http://www.virtu-als.com/}{games} and other
sorts of interactive edutainment) to train artificial intelligence.
Examples of anthropodidactic learning include using language textbooks
to train a machine translation model or using a flight simulator
developed for pilot training to train an autopilot with reinforcement
learning \href{https://openreview.net/pdf?id=H1mMHwt9X}{{[}Staudinger
Jorgensen Margineantu 2018{]}}.


\subsection{Why should anyone care?}\label{why-should-anyone-care}

Because the education industry puts a lot of effort into curating and
systematising knowledge in a way that can be useful for any learner,
whether they run on
\href{https://en.wikipedia.org/wiki/Human_brain}{carbon-} or
\href{https://en.wikipedia.org/wiki/Central_processing_unit}{silicon-based}
hardware. For example: - Exercise sets in mathematics, physics and
language learning, to name a few fields, are explicitly designed to
cover all important clusters/corner cases of the subject area -
something that isn't guaranteed in most datasets like logs, business
records or text corpora. - Exercise sets are also sorted by difficulty.
This creates a useful curriculum
\href{https://arxiv.org/abs/2101.10382}{{[}Soviany 2021{]}} to follow
when training a machine learning model. - Educational software aims to
give users immediate and precise feedback on their mistakes: delayed
gratification, as it turns out, is hard for
\href{https://en.wikipedia.org/wiki/Stanford_marshmallow_experiment}{people}
and reinforcement learning algorithms
\href{https://www.nowpublishers.com/article/Details/PGL-010}{{[}Gulwani
et al 2017{]}} alike.

See The Art of Problem Posing
\href{https://www.taylorfrancis.com/books/mono/10.4324/9781410611833/art-problem-posing-stephen-brown-marion-walter}{{[}Brown,
Walter 2004{]}} to learn more about\ldots{} the art of problem posing.


\subsection{This has been done already, hasn't it?}\label{this-has-been-done-already-hasnt-it}

Machine learning community is undoubtedly interested in taking lessons
from human learning, efforts to do so bear the umbrella term of
\href{https://ieeexplore.ieee.org/document/8481253}{antropomorphic
machine learning}. The prime example is curiculum learning
\href{https://arxiv.org/abs/2101.10382}{{[}Soviany 2021{]}}: it was born
with the observation that the order in which data is presented to human
students is crucial for them achieving their learning goals, so perhaps
it makes a difference for machines too.

However, examples of directly reusing learning aids developed for human
students are hard to come by. A notable exception is Reinforcement
Learning where decision-making agents are often trained on games
initially intended for people. And while the claim that
\href{https://gym.openai.com/envs/\#atari}{Atari games} and
\href{https://www.microsoft.com/en-us/research/project/project-malmo/}{Minecraft}
are educational material may be somewhat stretching the definition of
education, interactive simulators first developed for people and later
adapted for reinforcement learning include
\href{https://openreview.net/pdf?id=H1mMHwt9X}{X-plane} (used for
training pilots) and Virtu-ALS
\href{https://pubmed.ncbi.nlm.nih.gov/34977561/}{{[}Liventsev et al
2021{]}} (used for training nurses). Some antropodidactic work has also
been done in natural language processing, training language models on
children's books \href{https://arxiv.org/pdf/1511.02301.pdf}{{[}Hill et
al 2016{]}} and exercises for language learning
\href{https://aclanthology.org/2020.ngt-1.28.pdf}{{[}Mayhew et al
2020{]}} from \href{https://www.duolingo.com/}{Duolingo}

These examples, however, are exceptions rather than the rule and, in
general, anthropodidactic programming remains criminally underexplored.
A couple of research directions that seem extremely promising to me are:
- Training a language model on exercisebooks in subjects like
mathematics and the sciences to achieve a system capable of problem
solving in these fields. - Using beginner-level programming tasks to
develop program synthesis.

Didactic materials are a large class of useful data waiting for someone
to turn them into a successful artificial intelligence system/product.
Will it be \emph{you}, \%username\%?