\section{Programming by Example in Healthcare}


An Aritificial Intelligence PhD and a Healthcare PhD walk into a bar. The Healthcare PhD starts telling the AI PhD tales from the trenches of clinical research.

\subsection{Tale 1: Patient parsing}

The UVA/PADOVA Type 1 Diabetes Simulator](https://www.ncbi.nlm.nih.gov/pmc/articles/PMC4454102/) is an example of a mathematical smulator used for [in sillico trials](https://pubmed.ncbi.nlm.nih.gov/29868882/) of diabetes treatments. The researchers assessed [a dataset of diabetes patients collected in a bihormonal closed-loop clinical trial\footnote{\url{https://pubmed.ncbi.nlm.nih.gov/6390972/}} and developed a set of equations that describe these training data. These equations have become the standard tool to forecast how a certain treatment affects a diabetes patient.

\subsection{Tale 2: Doctor parsing}

Workaround studies](https://link.springer.com/content/pdf/10.1186/1472-6963-13-175.pdf) in Healthcare explore the deviations between prescribed [clinical protocols\footnote{\url{https://europepmc.org/article/med/7703111) and what doctors and nurses end up actually doing in real life practice. The reasons for these workarounds as well as their desirablitily are debated, but be they caused by individual clinicans' failures, organizational issues like understaffing and unrealistic expectations with regard to speed, poor quality of protocols themselves or (most likely}} all of the above, workarounds have to be studied and understood. Such research can shed more light on the ongoing debate, as well as on particular protocols in question. To that end, researchers reverse engineer the de facto protocols that clinicans follow.

\subsection{Programming by example}

"So", drawls the AI PhD, examining the bottom of his old fashioned glass, "these researchers are like a *homo sapiens* programming by example](https://www.microsoft.com/en-us/research/wp-content/uploads/2017/10/aplas17.pdf) system?" AI PhD hiccups and looks up from the glass to give the Healthcare PhD a glance of bemusement "You have a dataset of [trajectories](https://ai.stackexchange.com/questions/7359/what-is-a-trajectory-in-reinforcement-learning) and you are tasked with writing a protocol (a program) that can produce these trajectories. You write it, you test that it's consistent with the data and you publish it. You know we have", AI PhD pauses for another hearty hiccup, "[apps\footnote{\url{https://ml4code.github.io/}} for that, right?"

AI PhD is, at this point, too intoxicated to explain that programming by example
- studies automatic generation of programs based on a specification of input-output examples like those you would find at the bottom of a competitive programming problem statement\footnote{\url{https://codeforces.com/problemset/problem/1611/G}}
- has experienced explosive growth in recent years
- unlike other machine learning models, represents the induced algorithm as a program in a human-readable programming language, an approach that has a great potential in fields reluctant to adopt black box models\footnote{\url{https://www.nature.com/articles/s41562-021-01146-0}}

\subsection{Programming by example in Healthcare: a modest proposal}

This post is a proposal for a new field of research at the intersection of Program Synthesis, Process Mining, Imitation Learning and Healthcare. _Doctor parsing_ and _patient parsing_ are tasks that can and should be solved algorithmically by applying PbE techniques to datasets like MIMIC-IV\footnote{\url{https://mimic.mit.edu/}}. Healthcare is, after all, one of the most exciting applications of Program Synthesis.

Feel free to click around this diagram to learn more:

<div class="mxgraph" style="max-width:100%;border:1px solid transparent;" data-mxgraph="{&quot;highlight&quot;:&quot;#0000ff&quot;,&quot;nav&quot;:true,&quot;resize&quot;:true,&quot;toolbar&quot;:&quot;zoom layers tags lightbox&quot;,&quot;edit&quot;:&quot;_blank&quot;,&quot;xml&quot;:&quot;&lt;mxfile host=\&quot;app.diagrams.net\&quot; modified=\&quot;2021-12-02T14:38:12.333Z\&quot; agent=\&quot;5.0 (X11; Linux x86_64) AppleWebKit/537.36 (KHTML, like Gecko) Chrome/94.0.4606.54 Safari/537.36\&quot; etag=\&quot;bcjZ-Bvr9KBbh2TQkI_y\&quot; version=\&quot;15.8.2\&quot; type=\&quot;device\&quot;&gt;&lt;diagram id=\&quot;3228e29e-7158-1315-38df-8450db1d8a1d\&quot; name=\&quot;Page-1\&quot;&gt;3VpZd9o6EP41PJrj3eaRzUma0i1Jm/YlR9jCVrAtV5JDyK+/I2wHHOxbSICbXjjHy2if75vRaKBjDJPHM4ayaEIDHHd0NXjsGKOOrmumasNNSpaFxJFvUhAyEpSV1oIr8oRLoVpKcxJgXqsoKI0FyepCn6Yp9kVNhhiji3q1GY3ro2YoxFuCKx/F29IfJBBRKdXs3rrgHJMwKod2dacomCJ/HjKap+V4KU1xUZKgqptyjTxCAV1siIxxxxgySkXxlDwOcSzVWmmsaOe1lD5PmeFU7NJAnQScsGByY40/3C6v7qb53WdF1yyrhOoBxXmpjnLCYlnpZ7VCLDtSO8ZgERGBrzLky9IFMAJkkUhieNPgcUZTUUKsufKdxPGQxpSt+oIuBoOxA3IKPRAhGePIXrlgdI43KvZsx0D2c0kFjAGScsaYCfzYqg7tWcnAW0wTLNgSqlQN3BKXkrLV62IDf7uURRvQGxWeqORc+Nz1WvnwUOp/TyzU02LR84amN2xSvuGafdtoQ+nIWGiW27W24DCMBjg0W6uqHgUQ7bSAuIOx6w52MI6x43mj8dEAeYGHvW0czWisgTsKGsZp0fC88dCpG4HdiEbg9Kbq0cxDr6NhboNhug1g6OoRPZXpNEBhx6JUag0T+3dOqwKFr9Tdhwqggsd14Wp3rxS6Fhqz2dRQjU2RHcr7OUaxiHxQYjUuLKUYuqiwxQwAQNQx1014RzEJU3jxAQ0Mww8kVgRig35ZkJAgkH28JE6NKPJzYGf4p53IPCa8biu8SE6D4VkNqEgIGar15cC6t1gsugKlAWgpJinu+jQBaUAJXNGUw1VTuxBdGfLJsl0wbNW9NWzVVgERz0cc3QmwI6n9YeCbkwfq3C8v+vLT0S2j/+HXGe/b14vbnjm5Dc/w7EwZ9x1n8jW47fGR2/OuH0bUm2XG/HJKLuaj2WLyaGQxebr89eWuj80x/z4hP+6HZ0i/h0WgRPqF4rrNtBe8bmSphuS3xuY/8l9z6/wvR7MGE7QEIuoqlsMhgGu4shbV7VijVrI/i9H/xgD2jMycg+w9QH6j3bftSP6E+IxyOhMl9XGq5JL2DHOMmB/JapkCpxkhdaJ7eRZTFMgaQAtbWoWccDbFmt3NgtlBeLWnd37Zy668L0bLGA0ZShKShh25fRVjT1lVPpV440cwubjdf+9O6S0K78DUzb3f/jfWt/C3geU7U7ppD2/ktHsAHy8Zbb6F0UHcRX7SpSxs8uImGJ1nqJbZA9K+cN433sCbzodzNvIL5230k6spvh8GP39qs+v00yV2Jw6KzucZWi5vb0ZX5uff19HE0+962az3xTmLyDy861sL6/77g3cZTbJz2/z4/Wu6SLT5+zUL21ZV398ejSREIEFoCg1j8ATpyj7+Bo9+UPqb6glDHEl/6y30T2LTpwHuhkRE+bRL6Ep8GiK5+syRxtviX+VAy1REmBP+3nnUdjR6G7Ne+NXeCf1qa+zclDP4A9tq0QOA62POYfME91C63ffr61TVtpt8XbkKaFYs5L3z8wCRq2G+Rzq27/2vJcJbzkUlOWAXXIW+Urt8gx3rEDGgvqBsu0YLf9YN98wHbFCjntdZ5fLrOaFStDvvmvJP9QzVCQ9STWm8k9PRegUdZRJBqSa4ImQ7G3dNT8yUrApp4VRGcwbOCp4hfMVc5MFSnsoAJsZlIgPopyA/FxjObIymiC0V2HcDBmpRAsL9CLEQc4WkClIiyjOI8mKFYyEK3u7juc3Xmpq6+jQmFD5KpXyjHFWpBDiS7ZlKaDOyV83o8FD5KPUxAyxQmFIOtqgEOEZLheEg94uA+z9H4Rt9IikSFQba3umcnR1dIeYZSt++2AOCVEYEShEPKAgmgWkeK+nqSAQWk6cEHB4nYqlMwfgQVKqs6eBoDL5cQKuPaArXm08XV9fPuJivwwVkhcYPtC/9bcGOZu24u/SOubu86aS3Cr4jEkd5glLexblM3XHGi5ti7HLw2ysMSqQ+76nkYIC5z0hW+qq3RsrvIKip/a7WQEP54+WLXzkPwEr9lSHPK0gpPd3zvzdWZRv/jjHG/wA=&lt;/diagram&gt;&lt;/mxfile&gt;&quot;}"></div>
<script type="text/javascript" src="https://viewer.diagrams.net/js/viewer-static.min.js"></script>