Automatic program synthesis has been a long term goal of the field of artificial intelligence since its inception \cite{mannaAutomaticProgramSynthesis1971}, promising to reduce the workload of software developers by automatically solving some of the tasks they face.
And since the field's inception it has been grappling with the challenging properties of the sparse optimization space that is the set of all programs in a certain programming language, namely, 
\begin{itemize}
    \item valid error-free programs constitute an exceedingly small part of the space of possible strings, and
    \item a small change to 
\end{itemize}

\section{Deductive methods}

\section{Constrained problem spaces}

\subsection{Inductive logic programming}

\subsection{Domain specific languages}

\section{Human in the loop}

\paragraph{Completion}

\paragraph{Sketching}

\section{Evolutionary methods}

\section{Pre-training methods}

\subsection{Autoencoder models}

\subsection{Autoregressive models}

\section{Open problems}

However, a fully autonomous system for Programmatically Interpretable Reinforcement Learning is yet to be devised. In chapter \ref{ch:bfpp}