\chapter{Evaluating PatientSPIRL}
\label{ch:eval}

\section{Selecting the simulator}


\begin{table}[]
    \centering
    \begin{tabular}{c|c|c|c}
         Simulator & $\mathcal{O} $ & $\mathcal{A}$ & $p_r(r=0|s) \neq 1$ \\
         \midrule
         simglucose & $[0;+\infty)$ & $[0;35]$ & $S_{nt} \cup S_t$ \\
         GYMIC & $[0;24]^{46}$ & $0,\dots,24$ & $S_{nt} \cup S_t$ \\
         Virtu-ALS & $[0;256]^{307200}$ & $1,\dots,307200$ & $S_{nt} \cup S_t$ \\
         Auto-ALS & $[0;+\infty)^{36}$ & $1,\dots,34$ & $S_{nt} \cup S_t$ \\
         HeartPole & $\mathcal{R}^6$ & $1,\dots,4$ & $S_{nt} \cup S_t$ \\
         GraphSim & $(-\infty;+\infty)^{26}$ & $[0;1]^{317}$ & $S_t$ \\
    \end{tabular}
    \caption{Summary of each simulator: POMDP view}
    \label{tab:structview}
\end{table}

\begin{table*}[]
    \centering
    \begin{tabular}{c|c|c|c|c|c}
         Simulator & Scope & data source & sample size & learning algorithm & known biases \\
         \midrule
         simglucose \cite{simglucose} & type 1 diabetes & original study & 32 & expert model validated on data & \\
         GYMIC \cite{gymic} & sepsis in intensive care & MIMIC \cite{mimic} & 40000 & behavior cloning & overfitting \\
         Virtu-ALS \cite{briskAIEnhanceInteractive2018} & emergency care & & & & confirmation bias \\
         Auto-ALS & emergency care & & & & confirmation bias \\
         HeartPole & healthy lifestyle & & & & no factual basis \\
         GraphSim & intensive care & MIMIC \cite{mimic} & 40000 & graph compression & \\
    \end{tabular}
    \caption{Summary of each simulator: trust view}
    \label{tab:trustview}
\end{table*}

How do these simulators fare with respect to \emph{accuracy} criteria we set out in the introduction? 
The factors that contribute to a simulator's accuracy are reviewed in table \ref{tab:trustview}.
\emph{GYMIC} and \emph{GraphSim} are the only simulators trained on a large dataset and GYMIC's accurracy has known overfitting issues.
\emph{GraphSim} is thus the most accurate of the simulators.

The most \emph{transparent} simulator is clearly \emph{HeartPole}.
It does not aim to model any real clinical scenario accurately, but it can be a useful development tool to help scrutinize reinforcement learning algorithms.

As far as \emph{difficulty} is concerned, \emph{HeartPole} \cite{heartpole}, \emph{simglucose} \cite{simglucose-baseline}, and \emph{GYMIC} \cite{gymic} are known to be solvable with relatively small models and standard reinforcement learning algorithms like DQN \cite{dqn}. 
Thus, the only simulators \emph{difficult} enough to be benchmarks for novel approaches are \emph{Virtu-ALS} and \emph{Auto-ALS} and \emph{Auto-ALS} is the more accessible of the two.

Table \ref{tab:structview} reviews the structural complexity of the simulators, a factor that directly contributes to \emph{difficulty}.
Note that \emph{Virtu-ALS} is an unusually high-dimensional environment.
As such, solving it is likely to require more parameters and longer training times.
\emph{GraphSim} is the only simulator that does not provide non-zero rewards in non-terminal states $S_{nt}$, making it harder for the agent to attribute the results of the episode to particular actions.
\emph{GYMIC} (see section \ref{sec:gymic}) solves this problem with an additional metric (SOFA score), but unlike \emph{GYMIC} \emph{GraphSim} covers a wide range of clinical conditions and there is no single health metric applicable to each.


Automatic discovery of clinical strategies is a nascent field of research that has a potential to considerably improve patient outcomes and become a new modus operandi in healthcare research.
The goal of this chapter is to provide a solid foundation for further development of this field with better patient simulators and better understanding thereof.
We have reviewed the state of the art in patient simulators, identified some of the problems the field is facing and proposed novel simulators to address them.
We believe that \emph{HeartPole} and \emph{Auto-ALS} can become new standard benchmarks for reinforcement learning in healthcare, while \emph{GraphSim} can become a stepping stone to improved patient outcomes in intensive care.

\section{Auto-ALS}

\todo{Write based on results in W\&B}

\lstinputlisting{listings/auto-als.py}

\section{HeartPole}
\todo{Turn this into actual text}

\begin{tabular}{llrr}
\toprule
model & avg_score & best_avg_score \\
\midrule
deepseek-coder & -149.952560 & 0.400517 \\
gpt-4o & -99.885775 & 0.348758 \\
\bottomrule
\end{tabular}

\lstinputlisting{listings/metanurse-deepseek.py}

\lstinputlisting{listings/metanurse-gpt.py}

for 2868 iterations
