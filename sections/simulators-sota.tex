\chapter{The State of Patient Simulators}
\label{ch:simulators-sota}
\todo{Insert the relevant parts of the simulators paper and edit}


\paragraph{Intensive Care databases: a nano-review}
\label{sec:datasets}

% Why intensive care is a good source of data
Intensive Care is the medical speciality that supports patients whose lives are in immediate danger.
As such, it requires robust real time monitoring of the patient's vital signs due to quickly identify potential deterioration \cite{Bailey2013trial, Blount2010Real, Bockholt2022Real, Dimitrios1999Distributed, Fried2000Some, Mao2012integrated, Prgomet2016Vital, Vincent2018Improving}.
Monitoring hardware creates a datastream of variables like heart rate and blood oxygenation and, as a result, Intensive Care stands to benefit more than other fields of Healthcare from integration of data-driven models \cite{nunezreizBigDataAnalysis2019}.

% Datasets
MIMIC IV \cite{johnsonMIMICIVFreelyAccessible2023}, AmsterdamUMCdb \cite{amsterdamumcdb-a}, HiRID \cite{yecheHiRIDICUBenchmarkComprehensiveMachine} and the eICU Collaborative Research Database \cite{pollard2018a} are databases of health records obtained from Intensive Care Units.
Unlike many machine learning datasets, they avoid setting a standard for the parameters of the task such as which variables of a sample are to be used as features and which are to be predicted by the model, which samples are to be used for training and which are the holdout set - even what is a sample (a patient? an ICU admission? a drug? a diagnosis?). This makes them flexible and suitable for a wide array of tasks, but presents a challenge when seeking to compare different studies \cite{mcdermottReproducibilityMachineLearning2021}.

Derivative benchmarks, such as the MIMIC Benchmark \cite{harutyunyanMultitaskLearningBenchmarking2019} and HiRID-ICU-Benchmark \cite{yecheHiRIDICUBenchmarkComprehensiveMachine} seek to address the reproducibility issue: they put forth multiple datasets (one for each task) where all the information is derived from MIMIC and HiRID-ICU respectively, but is arranged specifically for the learning task at hand. They aim to become standard benchmarks for the tasks of mortality and length of stay prediction, patient fenotyping, prediction of circulatory, respiratory or kidney failure.

\paragraph{Towards foundation models for Healthcare}

Generalist models have demonstrated a superior performance to task specific models in many areas of machine learning \cite{reedGeneralistAgent2022} due to their ability to exploit implicit shared subtasks. 
This finding has precipitated the birth of a new paradigm known as foundation models \cite{zhouComprehensiveSurveyPretrained2023} - models trained on an all-encompassing dataset (such as the dataset that attempts to approximate all written text \cite{chelbaOneBillionWord2013}) and designed to be adapted to a broad array of specific downstream tasks.

The learning tasks typically studied in Healthcare are often interrelated. 
To use an example from section \ref{sec:datasets}, sepsis presents a high risk of death \cite{schlichtingRecognizingManagingSevere2007} and predicting sepsis is evidently useful for predicting mortality.
In light of existing research on generalist models, it is likely that considering them separately is counterproductive.
This, together with the vital societal importance of access to healthcare amidst understaffing \cite{ashleyy.metcalfHospitalUnitUnderstaffing2016, hudsonUnderstaffing2015, mercerMessageEditorinChief2008, munnUnderstaffingWardsCompromising2017, r.stanleyUnderstaffedOverwhelmed2010, SurveyShowsHidden1993, thelancetHealthcareSystemStaffing2018, UnderstaffingSignificantIssue2012} and population ageing \cite{2012health, Aslam2021Ageing, L1991aging, Lloyd2012Population, Mahishale2015Ageing, Mann2004aging, Sammy2019global, Suzman2015Health}, makes a foundation model for Healthcare a particularly important research goal.